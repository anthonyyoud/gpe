% $Id$
%------------------------------------------------------------------------------

\begin{chapter}{\label{cha:introduction}Introduction}
  The GPE code is a modular 3D Gross-Pitaevskii equation solver, written in
  Fortran 90 and parallelised using the Message Passing Interface (MPI).
  
  The code can solve a range of problems for both homogeneous and
  non-homogeneous (trapped) condensates, including vortex dynamics (lines,
  rings, rarefaction pulses), and non-equilibrium dynamics (condensate
  formation).

  Time stepping is performed explicitly using one of the following methods:
  %
  \begin{itemize}
    \item first-order Euler (E1);
    \item second-order Runge--Kutta (RK2);
    \item fourth-order Runge--Kutta (RK4);
    \item fourth/fifth-order adaptive Runge--Kutta--Fehlberg (RK45).
  \end{itemize}
  %
  The RK45 scheme uses the algorithm proposed in Numerical Recipes
  \citep[\S 16.2, p.708,][]{NR92}.
  
  The spatial discretisation is performed with either second- or fourth-order
  accurate centred finite differences, and the boundary conditions can be
  either periodic or reflective.

  The ``kind'' of floating point variables is parametrised, so real or
  double-precision arithmetic can be specified without relying on compiler
  switches.

  \section{\label{sec:prelim}Preliminaries and required software}
  Since the code uses MPI for the parallelism, an MPI parallel environment must
  be available (currently, this is also true even when running on one
  processor).  The code has been tested with the MPICH and OpenMPI
  implementations of MPI, but other implementations which adhere to the MPI
  standard should also be useable.

  The code uses some Fortran 2003 features including allocatable arrays within
  user-defined types, the \verb"protected" keyword, and stream I/O, so a
  compiler which allows some Fortran 2003 constructs is required.  Compilers
  known to work include sunf95 (Sun Studio 12), ifort (Intel Fortran Compiler),
  and gfortran (GNU Fortran Compiler).

  The code also requires the FFTw library (http://www.fftw.org) for some
  routines.  This \textbf{must} be major version 2 of the library (any minor
  version will be sufficient --- the latest is version 2.1.3), since the FFT is
  performed in parallel, and FFTw version 3 does not yet support (distributed)
  parallel transforms.

  To view the output of the code, any graphics program capable of reading
  space-separated columnar text files will be able to produce time-series plots
  (\eg gnuplot, http://www.gnuplot.info/).  Routines written in IDL
  (Interactive Data Language, http://www.ittvis.com/ProductServices/IDL.aspx)
  for producing 2D contour and 3D isosurface plots are included with the code.
  The data for these plots are saved in binary format, so more work will be
  needed if a different graphics program is to be used.  Volume renderings
  using VAPOR (http://www.vapor.ucar.edu/) are also possible.

  \subsection{Summary of required hardware and software}
  The following outlines the required hardware and software in order to run the
  code.  Other hardware and software may work, but has not been tested.
  %
  \begin{itemize}
    \item A Unix-like operating system, running on x86 or x86\_64 hardware.
    \item A standard development environment including make.
    \item A Fortran 90 compiler, supporting Fortran 2003 constructs (see
      above).
    \item An implementation of MPI, \eg OpenMPI or MPICH.
    \item Version 2.x.y of the FFTw library.
    \item gnuplot and/or IDL for visualisation; VAPOR optional.
  \end{itemize}

  \section{Organisation of this manual}
  The rest of this manual is organised as follows:
  %
  \begin{itemize}
    \item Chapter~\ref{cha:equations} ---
  \end{itemize}
\end{chapter}
