% $Id$
%------------------------------------------------------------------------------

\begin{chapter}{\label{cha:quickstart} Quick start guide}
  \section{Introduction}
  This chapter is intended as a quick start guide to help you compile, run and
  view the results of the GPE code.  More detailed information about various
  technical aspects of the code can be found in chapter SOMETHING OR OTHER;
  here we will outline the basics of going through a run cycle.

  \section{Setting parameters and compilation}
  There are three files you will need to edit to set up a run.  These are
  \verb"parameters.in", \verb"run.in", and \verb"ic.in".  The first two files
  set the run-time parameters, and the third sets the initial condition.

  A recompilation of the code is necessary only when changing the parameters in
  \verb"parameters.in", or the initial condition in \verb"ic.in".
  
  Set the parameters and initial condition as you wish.  All the
  parameters are explained in chapter SOMETHING OR OTHER.

  Choose a directory where any runs will take place.  This should not be on a
  network file system (NFS), which will likely be too slow (and under quota
  restrictions).

  Usually, it is a good idea to have a directory structure which is quite
  descriptive of the runs you are doing, so, for example, maybe you are doing
  runs where the initial conditions are different, or for a particular initial
  condition the parameters are varying.

  In this case you might create directories such as \verb"ring", or
  \verb"lines", with subdirectories defining ring radii or line separations,
  etc.  Ultimately, it is up to you, but try to keep things organised,
  otherwise it can be difficult to keep track of what you have done.

  Now compile the code with
  %
  \begin{Verbatim}
    ./setup run_dir
  \end{Verbatim}
  %
  This will compile the code, create the directory \verb"run_dir", and copy (or
  move) certain files to this directory.

  Now change to the run directory and type \verb"ls" to see its contents.

  In here you will find three files initially:
  %
  \begin{itemize}
    \item \verb"parameters.f90": the parameters file is copied from the source
      directory so that you know what parameters are related to the run;
    \item \verb"gpe": the executable which you will run;
    \item \verb"run.sh": a run script which sets up the run directory and runs
      the code.
  \end{itemize}
  %
  \section{Running the code}
  To run the code simply type
  %
  \begin{Verbatim}
    ./run.sh <nprocs>
  \end{Verbatim}
  %
  where \verb"<nprocs>" specifies on how many processes the code should run.
  Ideally, this should be less than or equal to the number of physical
  processors in your system, and it should match \verb"nyprocs*nzprocs" in
  \verb"parameters.f90", otherwise a run-time error will result.

  \section{During the run}
  When the job is running there is an empty file called \verb"RUNNING" in the
  run directory; when the job is finished this file is deleted.  You can also
  delete this file at any time to immediately stop the run cleanly.

  Any output that the job would normally print to the screen is instead
  redirected to the file \verb"log.txt".  It is a good idea to periodically
  check this file to make sure that no unexpected run-time errors have
  occurred.

  There are various output files containing various data, most of which are
  explained in chapter SOMETHING OR OTHER.

  The numbered \verb"proc" directories contain data local to each process on
  which the job ran.  This is where 3D isosurface data are stored.

  \section{Graphics}
  During and after a run it is possible to plot various 1D graphs, 2D contour
  plots and 3D isosurfaces.  For the 1D time-series graphs, gnuplot is a good
  program to use.  So, for example, to plot the line length of vortices in the
  condensate you could do
  %
  \begin{Verbatim}
    p "linelength.dat" u 1:2 w lp
  \end{Verbatim}
  %
  which will plot column two versus column one in the file, and where the
  \verb"plot", \verb"using", \verb"with", and \verb"linespoints" keywords have
  been abbreviated to the shortest non-ambiguous form, which is possible with
  any gnuplot command.

  \subsection{Contour plots}
  Contour and isosurface data are stored in binary form; the layout of the
  binary files are explained in chapter SOMETHING OR OTHER.  Assuming you have
  access to the Interactive Data Language (IDL) graphics program, IDL routines
  are provided with the code to directly plot the data.

  From the run directory start IDL.  Providing everything is set up correctly
  you should be able to type
  %
  \begin{Verbatim}
    anim,0,0,/cntr
  \end{Verbatim}
  %
  which will show a contour plot of the initial condition as a slice through
  the $(x,y)$-plane at $z=0$.

  Now try a plot at a later time replacing the zeroes in the previous command
  with $430$.  You should see that the two black patches have moved further to
  the right.

  The word \verb"anim" represents an IDL program, which takes two non-optional
  arguments.  The first is the index corresponding to the first file you want
  to plot, and the second is the index corresponding to the last file you want
  to plot.  \emph{Important: For output to the screen these numbers should
  always be the same.}

  The numbers refer to the files within the process directories.  If you look
  in \verb"proc00", for example, you will see a number of files named
  \verb"dens???????.dat".  The arguments to \verb"anim" are the numbers within
  the filenames excluding the leading zeroes.

  So \verb"anim,0,0" plots \verb"dens0000000.dat"; \verb"anim,331,331" plots
  \verb"dens0000331.dat", \etc

  The \verb"/cntr" option is a switch which turns on contour output, rather
  than the default which is an isosurface (see below).

  To quite IDL type \verb"exit".

  \subsection{Isosurface plots}
  %
  Start IDL again, and do exactly as before, but this time leave off the
  \verb"/cntr" switch.  You should see a white box with nothing in it, and a
  control panel with various buttons on it.  Click \verb"bbox" and
  \verb"content" and then press \verb"Redraw".  You should now see an
  isosurface of the vortex ring.  Left-click and hold in the white window and
  you can move the view around; right-click zooms in and out, and middle-click
  moves the centre viewpoint.  Zooming and moving the view might be a little
  slow, so move the mouse slowly, and avoid large jerky movements.

  The coloured bar in the control panel is a histogram of the density, and
  clicking in here will redraw the isosurface at the new density level, which
  is shown toward the right-hand side of the control panel.

  Pressing \verb"auto" will automatically redraw the isosurface when other
  boxes are checked, for example, the axes, or whether the surface is solid,
  wireframe, or a series of points.  If the display is too slow, selecting
  \verb"points" or \verb"wireframe" instead of \verb"solid" will speed it up.

  \subsection{Animations}
  %
  By saving a series of snapshots it is possible to then combine them into an
  animation.

  From the run directory run the script \verb"rename.sh".  This will create a
  directory \verb"links", which contains renamed symbolic links (shortcuts)
  pointing to the isosurface files within each process directory.  This is to
  make sure that the files are numbered sequentially in steps of one.

  \subsubsection{Contour animations}
  %
  Change to the links directory and create another directory called
  \verb"images", then start IDL again.

  Now type
  %
  \begin{Verbatim}
    anim,0,9,/cntr,/c_anim
  \end{Verbatim}
  %
  This will loop over all data files and save \verb"dens???????.png" files in
  the \verb"images" directory.

  \emph{If you miss off the /c\_anim switch here, you will get all the
  plots shown to the screen.  When you are plotting $100$ or more files, this
  is very bad and will run the machine out of memory!}

  Now change to the \verb"images" directory and run
  %
  \begin{Verbatim}
    makemovie -i png -p dens
  \end{Verbatim}
  %
  which will create an AVI animation out of the PNG files, with $10$ frames,
  saving the output as \verb"output.avi".

  Play the animation with
  %
  \begin{Verbatim}
    mplayer output.avi
  \end{Verbatim}
  %
  Strangely, this is now in colour!  The animation is very short, but it gives
  you an idea of what you can do.

  \subsubsection{Isosurface animations}
  %
  Rename the \verb"images" directory to something else, for example,
  %
  \begin{Verbatim}
    mv images contour
  \end{Verbatim}
  %
  Now recreate an empty \verb"images" directory.  Start IDL and this time type
  %
  \begin{Verbatim}
    anim,0,9,/png
  \end{Verbatim}
  %
  As before this will save PNG files in the \verb"images" directory, which you
  can then convert and play with \verb"mencoder" and \verb"mplayer".

  \emph{If you miss off the /png switch, all the output will go to the screen.
  Avoid!}

  \section{Other information}
  %
  It is possible for the code to restart itself in the event of a computer
  crash, if the \verb"RUNNING" file exists in the run directory (if it doesn't
  you can just create a new empty file).  If you ever need to do this, just let
  me know.

  It is also possible to do a manual restart of the code after a run has
  cleanly ended.  The usual reason for wanting to do is to extend the length of
  the run, without having to start from the beginning again.  I will show you
  how to do this as and when you need to.

  If there is anything that is unclear, or you want to do something that the
  code doesn't do, then just let me know.
\end{chapter}
