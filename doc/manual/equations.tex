%
% Copyright (c) Anthony J. Youd/Newcastle University 2011
%
\begin{chapter}{\label{cha:equations}Governing equations and
  non-dimensionalisation}

  The single-particle complex wavefunction $\psi(\vec{r}, t)$ for $N$ bosons of
  mass $m$, obeys the 3-dimensional, time dependent, Gross--Pitaevskii (GP)
  equation \citep{Gross61,Pitaevskii61}
  %
  \begin{equation}
    \eye \hbar \frac{\pa \psi}{\pa t} = -\frac{\hbar^{2}}{2m} \nabla^{2} \psi +
    \Vtrap \psi + g \abs{\psi}^{2} \psi - \mu \psi,
    \label{eqn:gpe}
  \end{equation}
  %
  where $\hbar = h/(2\pi)$ is the reduced Planck constant, $\Vtrap$ is an
  external trapping potential, $g$ is the strength of the interactions between
  the bosons, and $\mu$ is the chemical potential.  The wavefunction is
  normalised by the condition that
  %
  \begin{equation*}
    \int_{V}{\abs{\psi}^{2}}\total V = N.
  \end{equation*}
  %
  The GPE describes the evolution of the ground state of a quantum system of
  weakly interacting bosons, in the limit of zero temperature, and when the
  number of bosons $N$ is large.

  \section{Imaginary time}
  In making the transformation $t \to -\eye t$, equation~\eqref{eqn:gpe}
  becomes
  %
  \begin{equation*}
    -\hbar \frac{\pa \psi}{\pa t} = -\frac{\hbar^{2}}{2m} \nabla^{2} \psi +
    \Vtrap \psi + g \abs{\psi}^{2} \psi - \mu \psi,
  \end{equation*}
  %
  which can be thought of as a modified diffusion equation.  For certain
  initial conditions, it is sometimes necessary to run the code in imaginary
  time, because the initial conditions are not exact solutions of the GPE.  By
  propagating in imaginary time, these solutions tend to the ground state,
  which is an exact solution of the GPE.  The code can then be propagated in
  real time.

  To propagate in imaginary time set \verb"real_time" equal to \verb".true.";
  otherwise set \verb"real_time" equal to \verb".false.".

  \section{\label{sec:nondimgpe}Non-dimensionalised GPE}
  The code solves a non-dimensionalised form of equation~\eqref{eqn:gpe}.
  There are two possibilities for this non-dimensionalisation, depending on
  whether the external trapping potential is present.  Here, we show the two
  cases.

  \subsection{Natural units}
  When the external trapping potential is absent, natural units are used to
  non-dimensionalise the GPE.  In this case the scalings are
  %
  \begin{equation}
    \begin{aligned}
      t       &\to \frac{\hbar}{2\mu}t, \\
      \vec{r} &\to a\vec{r}, \\
      \psi    &\to \psi_{\infty} \psi,
    \end{aligned}
    \label{eqn:natural_scalings}
  \end{equation}
  %
  where $a = \hbar/(\sqrt{2m\mu})$ is the healing length, and $\psi_{\infty} =
  \sqrt{\mu/g}$ is the bulk value of $\psi$.

  These scalings lead to the dimensionless form of the GPE
  %
  \begin{equation*}
    -2\eye\frac{\pa \psi}{\pa t} = \nabla^{2}\psi + \left(1 -
    \abs{\psi}^{2}\right)\psi.
  \end{equation*}
  %
  See appendix~\ref{cha:nondim} for a full derivation.  To solve this form of
  the GPE, set \gpevar{eqn\_to\_solve} to \gpevar{1} in \gpefile{run.in}.

  \subsection{Harmonic oscillator units}
  When the external trapping potential is present, harmonic oscillator units
  are used to non-dimensionalise the GPE.  The scalings are
  %
  \begin{equation}
    \begin{aligned}
      t       &\to \frac{t}{\ombar}, \\
      \vec{r} &\to \aoh\vec{r}, \\
      \psi    &\to \aoh^{-\frac{3}{2}} \psi, \\
      g       &\to \aoh^{3}\hbar\ombar g, \\
      \mu     &\to \hbar\ombar\mu, \\
      \Vtrap  &\to \hbar\ombar\Vtrap,
    \end{aligned}
    \label{eqn:ho_scalings}
  \end{equation}
  %
  where $\ombar = \left( \omega_{x}\omega_{y}\omega_{z}
  \right)^{1/3}$, $\omega_{i}$ is the trap frequency along axis $i$, for $i =
  x$, $y$, $z$, and $\aoh = \sqrt{\hbar/(m\ombar)}$ is the
  harmonic oscillator length.

  These scalings lead to the alternative dimensionless form of the GPE
  %
  \begin{equation*}
    \eye \frac{\pa \psi}{\pa t} = -\frac{1}{2}\nabla^{2}\psi + \Vtrap\psi +
    g\abs{\psi}^{2}\psi - \mu\psi.
  \end{equation*}
  %
  See appendix~\ref{cha:nondim} for a full derivation.  To solve this form of
  the GPE, set \gpevar{eqn\_to\_solve} to \gpevar{4} in \gpefile{run.in}.

  \subsection{Alternative form}
  An alternative dimensionless form of the GPE can also be solved, specifically
  for the random phase approximation, as in \citet{BS02}.  This is
  %
  \begin{equation*}
    -2\eye\frac{\pa \psi}{\pa t} = \nabla^{2}\psi + \abs{\psi}^{2}\psi,
  \end{equation*}
  %
  and it can be selected by setting \gpevar{eqn\_to\_solve} to \gpevar{2} in
  \gpefile{run.in}.

  \section{Dimensionless forms of other quantities}
  It is useful to know the dimensionless forms of other relevant variables and
  quantities.  These are described in this section.  Full derivations of each
  of the non-dimensionalisations are given in appendix~\ref{cha:nondim}.

  \subsection{Harmonic trapping potential}
  The dimensional harmonic trapping potential is given by
  %
  \begin{equation}
    \Vtrap = \frac{1}{2} m \left( \omega_{x}^{2} x^{2} + \omega_{y}^{2} y^{2} +
    \omega_{z}^{2} z^{2} \right).
    \label{eqn:trap}
  \end{equation}
  %
  Using the scaling $\omega_{i} \to \ombar \omega_{i}$, for $i=x$, $y$, $z$,
  and remembering to scale each coordinate, leads to the dimensionless form of
  the trapping potential
  %
  \begin{equation*}
    \Vtrap = \frac{1}{2} \left( \omega_{x}^{2} x^{2} + \omega_{y}^{2} y^{2} +
    \omega_{z}^{2} z^{2} \right).
  \end{equation*}

  \subsection{Thomas--Fermi approximation}
  The dimensional Thomas--Fermi approximation is given by
  %
  \begin{equation*}
    \psi = \sqrt{\frac{\mu-\Vtrap}{g}}.
  \end{equation*}
  %
  Since this approximation is only relevant for trapped condensates, we use
  harmonic oscillator units to non-dimensionalise.  This leads to an identical
  dimensionless expression, where each dimensional variable is replaced by its
  dimensionless counterpart.

  \subsubsection{Condensate extent}
  The extent of the condensate $R_{i}$, $i = x$, $y$, $z$, in the Thomas--Fermi
  limit, is given by
  %
  \begin{equation*}
    R_{i}^{2} = \frac{2\mu}{m\omega_{i}^{2}}, \quad i = x, y, z.
  \end{equation*}
  %
  Non-dimensionalising using harmonic oscillator units yields
  \begin{equation*}
    R_{i}^{2} = \frac{2\mu}{\omega_{i}^{2}}, \quad i = x, y, z.
  \end{equation*}

  \subsubsection{Number of atoms}
  The number of atoms within the condensate $N$, under the Thomas--Fermi
  approximation, is given by
  %
  \begin{equation*}
    N = \frac{8\pi}{15} \left( \frac{2\mu}{m\ombar^2} \right)^{\frac{3}{2}}
    \frac{\mu}{g}.
  \end{equation*}
  %
  Again, non-dimensionalising using harmonic oscillator units, leads to
  %
  \begin{equation*}
    N = \frac{16\sqrt{2}\pi}{15} \frac{\mu^{\frac{5}{2}}}{g}.
  \end{equation*}

  \subsection{Circulation}
  The dimensional circulation $\kappa$, around a vortex is defined to be
  %
  \begin{equation*}
    \kappa = \oint_{C}{\vec{u}\bcdot\total\vec{l}}.
  \end{equation*}
  %
  Then, using the fact that $\vec{u} = (\hbar/m)\bnabla\phi$, where $\phi$ is
  the phase, and also noting that the circulation is quantised, such that the
  phase differs by $2\pi n$ around the vortex, where $n$ is the winding number,
  we obtain
  %
  \begin{equation*}
    \kappa = \frac{\hbar}{m} \oint_{C}{\bnabla\phi\bcdot\total\vec{l}} =
    \frac{2\pi\hbar}{m}n.
  \end{equation*}
  %
  Using natural units with the scaling $\kappa \to (2\mu a^{2}/\hbar)\kappa$,
  or harmonic oscillator units with the scaling $\kappa \to
  \aoh^{2}\ombar\kappa$, in both cases leads to a dimensionless circulation of
  $\kappa = 2\pi n$.

  \section{Initial conditions}
  The code implements three basic initial conditions, any combination of which
  can theoretically be multiplied together.  This section outlines the
  mathematical description of the initial conditions; for a description of the
  numerical implementation see \S\ref{subsec:ic.in}.

  \subsection{Vortex line}
  A vortex line is given by
  %
  \begin{equation*}
    \psi_{0} = f(r) \exp{(\eye \theta)},
  \end{equation*}
  %
  in polar coordinates $(r, \theta)$, where $f(r)$ is some function which
  models the vortex core, and $\theta$ is the phase.  In the code, we use the
  vortex core model proposed by \citet{BR01}, so that
  %
  \begin{equation*}
    f(r) = \left[ 1 - \exp{\left( -0.7r^{1.15}\right)} \right] \exp{(\eye
    \theta)}.
  \end{equation*}
  %
  The vortex line initial condition can also support planar or helical
  perturbations along its length.  For a vortex line oriented along the
  $z$-direction, this is achieved by imposing a sinusoidal
  perturbation of the form
  %
  \begin{equation*}
    \sin{\left( \frac{2\pi z}{l} \right)},
  \end{equation*}
  %
  in the $x$-direction, and the same perturbation shifted by $\pi/2$ in the
  $y$-direction.  These perturbations can be set independently; doing so will
  result in a planar perturbation, while setting both will result in a helical
  perturbation.

  Cyclically permuting $x$, $y$, and $z$ results in a vortex line along another
  direction, and the initial position and circulation of the line can also be
  defined.  See \S\ref{subsec:parameters.in} for a description of the
  parameters which control the vortex line initial condition.

  \subsection{Vortex ring}
  A vortex ring in the $(y,z)$-plane, travelling in the $x$-direction takes the
  form \citep{Berloff04}
  %
  \begin{equation*}
    \psi_{0} = \Psi(x, s+R_{0}) \Psi^{*}(x, s-R_{0}),
  \end{equation*}
  %
  where $\Psi(x,s)$ is given by
  %
  \begin{equation*}
    \Psi(x,s) = f\left( \sqrt{x^{2}+s^{2}} \right) \exp{(\eye \theta)},
  \end{equation*}
  %
  $f(r)$ (which again models the vortex core) is given by
  %
  \begin{equation*}
    f(r)^{2} = \frac{r^{2}\left( a_{1} + a_{2}r^{2} \right)}{1 + b_{1}r^{2} +
    b_{2}r^{4}},
  \end{equation*}
  %
  where $a_{i}$ and $b_{i}$ are constants, $s = \sqrt{y^{2} + z^{2}}$, and
  $R_{0}$ is the radius of the ring.

  The vortex ring initial condition also supports planar and helical
  perturbations, in a manner similar to that of the vortex line.  The perturbed
  initial condition then takes the form
  %
  \begin{equation*}
    \begin{aligned}
    \psi_{0} = \Psi&\left\{ x-A_{1} \cos{\left( m_{1}\theta \right)},
    s+R_{0}-A_{2} \cos{\left( m_{2}\theta \right)} \right\} \times \\
    \Psi^{*}&\left\{ x-A_{1} \cos{\left( m_{1}\theta \right)},
    s-R_{0}-A_{2} \cos{\left( m_{2}\theta \right)} \right\},
    \end{aligned}
  \end{equation*}
  %
  where $\Psi(x,s)$ is given by
  %
  \begin{equation*}
    \Psi(x,s) = f\left( \sqrt{x^{2}+s^{2}} \right) \left( x+is \right).
  \end{equation*}
  %
  The function $f(r)$ is as above, $s = \sqrt{y^{2} + z^{2}} - A_{1}\sin{\left(
  m_{1}\theta \right)}$, $A_{1}$ and $m_{1}$ are the amplitude and wavenumber
  of a purely helical perturbation, and $A_{2}$ and $m_{2}$ are the amplitude
  and wavenumber of a purely planar perturbation.

  The direction of motion of the vortex ring in the $x$-direction can be
  defined, but note that it is not yet possible to set up rings moving in the
  $y$- or $z$-directions.  See \S\ref{subsec:parameters.in} for a description
  of the parameters which control the vortex ring initial condition.

  \subsection{Random phase}
  The GPE can model the formation of a condensate, starting from a
  non-equilibrium initial condition.  The random phase initial condition
  describes a weakly interacting Bose gas, where the particles remain in a
  strongly non-equilibrium state.  Evolving this state leads to the formation
  of a quasi-condensate, consisting of a tangle of quantised vortices.  This
  tangle decays as the system reaches thermal equilibrium, with a certain
  number of particles in the zero-momentum (genuine condensate) state.  See,
  for example, \citet{BS02}, \citet{CJPPR05}, and \citet{BY07} for more details
  on the random phase approximation.

  To model this, the initial condition is set to
  %
  \begin{equation*}
    \psi_{0} = \sum_{\vec{k}} a_{\vec{k}} \exp{(\eye\vec{k}\bcdot\vec{r})},
  \end{equation*}
  %
  where the $\vec{k}$s are the wavenumbers in momentum space, and the phases of
  the complex amplitudes $a_{\vec{k}}$ are distributed randomly.

  The mass and kinetic energy density must be set in \gpefile{run.in}, prior to
  performing a random phase simulation.  These are \verb"nv" and \verb"enerv"
  respectively.

\end{chapter}
