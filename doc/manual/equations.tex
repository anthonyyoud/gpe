% $Id: equations.tex 604 2010-06-09 11:24:10Z najy2 $
%------------------------------------------------------------------------------

\begin{chapter}{\label{cha:equations}Governing equations and
  non-dimensionalisation}

  The single-particle complex wavefunction $\psi(\vec{r}, t)$ for $N$ bosons of
  mass $m$, obeys the 3 dimensional, time dependent Gross--Pitaevskii (GP)
  equation \citep{Gross61,Pitaevskii61}
  %
  \begin{equation}
    \eye \hbar \frac{\pa \psi}{\pa t} = -\frac{\hbar^{2}}{2m} \nabla^{2} \psi +
    \Vtrap \psi + g \abs{\psi}^{2} \psi - \mu \psi,
    \label{eqn:gpe}
  \end{equation}
  %
  where $\hbar = h/(2\pi)$ is the reduced Planck constant , $\Vtrap$ is an
  external potential, $g$ is the strength of the interactions between the
  bosons, and $\mu$ is the chemical potential.  The wavefunction is normalised
  by the condition that
  %
  \begin{equation*}
    \int_{V}{\abs{\psi}^{2}}\total V = N.
  \end{equation*}
  %
  The GPE describes the evolution of the ground state of a quantum system of
  weakly interacting bosons, in the limit of zero temperature, and when the
  number of bosons $N$ is large.

  \section{Non-dimensionalised GPE}
  The code solves a non-dimensionalised form of equation~\eqref{eqn:gpe}.
  There are two possibilities for this non-dimensionalisation, depending on
  whether the external trapping potential is present.  Here, we show the two
  cases.

  \subsection{Natural units}
  When the external trapping potential is absent, natural units are used to
  non-dimensionalise the GPE.  In this case the scalings are
  %
  \begin{equation*}
    \begin{aligned}
      t       &\to \frac{\hbar}{2\mu}t \\
      \vec{r} &\to a\vec{r} \\
      \psi    &\to \psi_{\infty} \psi,
    \end{aligned}
  \end{equation*}
  %
  where $a = \hbar/(\sqrt{2m\mu})$ is the healing length and $\psi_{\infty} =
  \sqrt{\mu/g}$ is the bulk value of $\psi$.

  These scalings lead to the dimensionless form of the GPE
  %
  \begin{equation*}
    -2\eye\frac{\pa \psi}{\pa t} = \nabla^{2}\psi + \left(1 -
    \abs{\psi}^{2}\right)\psi.
  \end{equation*}
  %
  See appendix~\ref{cha:nondim} for a full derivation.

  \subsection{Harmonic oscillator units}
  When the external trapping potential is present, harmonic oscillator units
  are used to non-dimensionalise the GPE.  The scalings are
  %
  \begin{equation*}
    \begin{aligned}
      t       &\to \frac{t}{\ombar} \\
      \vec{r} &\to \aoh\vec{r} \\
      \psi    &\to \aoh^{-\frac{3}{2}} \psi \\
      g       &\to \aoh^{3}\hbar\ombar g \\
      \mu     &\to \hbar\ombar\mu \\
      \Vtrap  &\to \hbar\ombar\Vtrap,
    \end{aligned}
  \end{equation*}
  %
  where $\ombar = \left( \omega_{x}\omega_{y}\omega_{z}
  \right)^{1/3}$, $\omega_{i}$ is the trap frequency along axis $i$ for $i =
  x$, $y$, $z$, and $\aoh = \sqrt{\hbar/(m\ombar)}$ is the
  harmonic oscillator length.

  These scalings lead to the alternative dimensionless form of the GPE
  %
  \begin{equation*}
    \eye \frac{\pa \psi}{\pa t} = -\frac{1}{2}\nabla^{2}\psi + \Vtrap\psi +
    g\abs{\psi}^{2}\psi - \mu\psi.
  \end{equation*}
  %
  \section{Dimensionless forms of other quantities}
  It is useful to know the dimensionless forms of other relevant variables and
  quantities.  These are described in this section.  Full derivations of each
  of the non-dimensionalisations is given in appendix~\ref{cha:nondim}.

  \subsection{Thomas--Fermi approximation}
  The dimensional Thomas--Fermi approximation is given by
  %
  \begin{equation*}
    \psi = \sqrt{\frac{\mu-\Vtrap}{g}}
  \end{equation*}
  %
  Since this approximation is only relevant for trapped condensates, we use
  harmonic oscillator units to non-dimensionalise.  This leads to an identical
  dimensionless expression, where each dimensionless variable is replaced by
  its dimensionless counterpart.

  \subsubsection{Condensate extent}
  The extent $R_{i}$, $i = x$, $y$, $z$, of the condensate in the Thomas--Fermi
  limit is given by
  %
  \begin{equation*}
    R_{i}^{2} = \frac{2\mu}{m\omega_{i}^{2}}, \quad i = x, y, z.
  \end{equation*}
  %
  Non-dimensionalising using harmonic oscillator units yields
  \begin{equation*}
    R_{i}^{2} = \frac{2\mu}{\omega_{i}^{2}}, \quad i = x, y, z.
  \end{equation*}

  \subsubsection{Number of atoms}
  The number of atoms within the condensate, under the Thomas--Fermi
  approximation is given by
  %
  \begin{equation*}
    N = \frac{8\pi}{15} \left( \frac{2\mu}{m\ombar^2} \right)^{\frac{3}{2}}
    \frac{\mu}{g}.
  \end{equation*}
  %
  Again, non-dimensionalising using harmonic oscillator units, leads to
  %
  \begin{equation*}
    N = \frac{16\sqrt{2}\pi}{15} \frac{\mu^{\frac{5}{2}}}{g}.
  \end{equation*}

  \subsection{Circulation}
  The dimensional circulation $\kappa$, around a vortex is defined to be
  %
  \begin{equation*}
    \kappa = \oint_{C}{\vec{u}\bcdot\total\vec{l}}.
  \end{equation*}
  %
  Then, using the fact that $\vec{u} = (\hbar/m)\bnabla\phi$, where $\phi$ is
  the phase, and also noting that the circulation is quantised, such that the
  phase differs by $2\pi n$ around the vortex, where $n$ is the winding number,
  we obtain
  %
  \begin{equation*}
    \kappa = \frac{\hbar}{m} \oint_{C}{\bnabla\phi\bcdot\total\vec{l}} =
    \frac{2\pi\hbar}{m}n.
  \end{equation*}
  %
  Using natural units with the scaling $\kappa \to (2\mu a^{2}/\hbar)\kappa$,
  or harmonic oscillator units with the scaling $\kappa \to
  \aoh^{2}\ombar\kappa$, leads to a dimensionless circulation of
  $\kappa = 2\pi n$.
\end{chapter}
