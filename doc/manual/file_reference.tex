% $Id: file_reference.tex 627 2010-06-18 08:19:50Z najy2 $
%------------------------------------------------------------------------------

\begin{chapter}{\label{cha:file_reference}File reference}
  This chapter is intended as a reference for the main files associated with
  the code, such as the source files, input and output files, and the IDL
  \gpefile{gpe.pro} program.

  \section{Program source files}
  Table~\ref{tab:source_files} lists the \gpefile{.f90} source code files which
  make up the code, and describes their functionality.
  %
  \begin{table}[ht]
    \centering
    \begin{tabular}{lp{0.67\textwidth}}
      \hline
      File & Description \\
      \hline
      \gpevar{constants.f90} & This is a module which defines the constants
      needed for FFTw. \\
      %
      \gpefile{derivs.f90} & Routines to do with derivatives. \\
      %
      \gpefile{error.f90} & Error-handling routines. \\
      %
      \gpefile{gpe.f90} & Main program file. \\
      %
      \gpefile{ic.f90} & Routines to do with setting up the initial condition
      and general initialisation. \\
      %
      \gpefile{parameters.f90} & Compile-time parameters and global variables.
      \\
      %
      \gpefile{solve.f90} & Routines to do with actually solving the equation,
      for example, the time stepping algorithms are defined in this file. \\
      %
      \gpefile{variables.f90} & User-defined types, and other general routines
      to do with the equation variables. \\
      \hline\hline
    \end{tabular}
    \caption{\label{tab:source_files}Program source files which make up the GPE
      code.}
  \end{table}

  \section{Program input files}
  This section describes the \gpefile{.in} input files which will generally
  need to be edited to set up a run.

  \subsection{\label{subsec:parameters.in}parameters.in}
  Edit this file to set the floating-point precision, the number of processes,
  and the grid dimensions of the run.  If the initial condition consists of a
  vortex line or vortex ring, then the line/ring parameters can also be set
  here.  Note that any changes to this file will require a recompilation of the
  code.  See tables~\ref{tab:parameters.in}, \ref{tab:line_params}, and
  \ref{tab:ring_params} for a description of the parameters.
  %
  \begin{table}[ht]
    \centering
    \begin{tabular}{llp{0.67\textwidth}}
      \hline
      Parameter & Type & Description \\
      \hline
      \gpevar{pr} & integer & The precision of real variables is parametrised.
      Choose the desired line for either real or double precision. \\
      %
      \gpevar{nyprocs} & integer & The number of processes in the
      $y$-direction. \\
      %
      \gpevar{nzprocs} & integer & The number of processes in the
      $z$-direction. \\
      %
      \gpevar{nx} & integer & The number of grid points in the $x$-direction.
      \\
      %
      \gpevar{ny} & integer & The number of grid points in the $y$-direction.
      \\
      %
      \gpevar{nz} & integer & The number of grid points in the $z$-direction.
      \\
      \hline\hline
    \end{tabular}
    \caption{\label{tab:parameters.in}Compile-time parameters to set.}
  \end{table}
   
  There are no restrictions on the \gpevar{nyprocs} and \gpevar{nzprocs}
  parameters other than that they be $\geqslant 1$.  In general, it is
  recommended that \gpevar{nzprocs} $\geqslant$ \gpevar{nyprocs} for best
  performance, so that fewer non-contiguous data transfers are performed.  They
  can both be set to 1, in which case the job is simply run on one process (an
  MPI parallel environment is still required though).
  %
  \begin{table}[ht]
    \centering
    \begin{tabular}{llp{0.6\textwidth}}
      \hline
      Parameter & Type & Description \\
      \hline
      \gpevar{x0} & real & $x$-position of the line. \\
      %
      \gpevar{y0} & real & $y$-position of the line. \\
      %
      \gpevar{z0} & real & $z$-position of the line. \\
      %
      \gpevar{amp1} & real & Amplitude of a sinusoidal disturbance in one
      direction along the line. \\
      %
      \gpevar{amp2} & real & Amplitude of a sinusoidal disturbance in the other
      direction along the line. \\
      %
      \gpevar{ll} & real & The wavelength of the above disturbances. \\
      %
      \gpevar{sgn} & real & The sign of the argument of the line (\ie
      circulation direction). \\
      %
      \gpevar{dir} & character & The direction ($x$, $y$, or $z$) in which the
      line should extend. \\
      %
      \gpevar{imprint\_phase} & logical & Whether only the phase should be
      imprinted, \ie no vortex core should be modelled. \\
      \hline\hline
    \end{tabular}
    \caption{\label{tab:line_params}Compile-time parameters for a vortex line.}
  \end{table}
  %
  \begin{table}[ht]
    \centering
    \begin{tabular}{llp{0.67\textwidth}}
      \hline
      Parameter & Type & Description \\
      \hline
      \gpevar{x0} & real & $x$-position of the ring. \\
      %
      \gpevar{y0} & real & $y$-position of the ring. \\
      %
      \gpevar{z0} & real & $z$-position of the ring. \\
      %
      \gpevar{amp} & real & Amplitude of a planar disturbance around the ring.
      \\
      %
      \gpevar{mm} & integer & Wavenumber of a planar disturbance. \\
      %
      \gpevar{r1} & real & Amplitude of a helical disturbance around the ring.
      \\
      %
      \gpevar{kk} & integer & Wavenumber of a helical disturbance. \\
      %
      \gpevar{dir} & real & Ring propagation direction ($\pm 1$). \\
      \hline\hline
    \end{tabular}
    \caption{\label{tab:ring_params}Compile-time parameters for a vortex ring.}
  \end{table}

  New vortex lines and rings can be defined simply by copying an existing
  definition in \gpefile{parameters.in}, and renaming, so for example, the
  \gpeexample{ring} example defines
  %
  \begin{Verbatim}
    type (ring_param), parameter :: &
      vr1 = ring_param(0.0_pr, 0.0_pr, 0.0_pr, 10.0_pr, &
        0.0_pr, 5, 0.0_pr, 10, -1.0_pr)
  \end{Verbatim}
  %
  To create another vortex ring definition, say with a radius of $20$, and
  situated at $x=5$, define
  %
  \begin{Verbatim}
    type (ring_param), parameter :: &
      vr2 = ring_param(5.0_pr, 0.0_pr, 0.0_pr, 20.0_pr, &
        0.0_pr, 5, 0.0_pr, 10, -1.0_pr)
  \end{Verbatim}
  %
  Then an initial condition consisting of these two vortex rings could be set
  up as described in the next section. 

  \subsection{\label{subsec:ic.in}ic.in}
  This file defines the initial condition.  The initial condition must be
  defined in the form \gpevar{init\_cond = function()}, where
  \gpevar{function()} is some function in the code which defines a possible
  component of an initial condition.  Components can be multiplied together to
  form any number of different initial conditions.  Look at \gpefile{ic.in} in
  the main code directory to see some examples.

  In the \gpeexample{ring} example, the initial condition is set to
  \gpevar{vortex\_ring(vr1)}, where \gpevar{vr1} is a type parameter declared
  in \gpefile{parameters.in} (see above).

  As another example, if you define \gpevar{vr2} as above, then you could
  construct an initial condition consisting of two rings with
  %
  \begin{Verbatim}
    init_cond = vortex_ring(vr1) * vortex_ring(vr2)
  \end{Verbatim}
  %
  In this way, any number of initial conditions can be constructed, simply by
  multiplying functions together.

  As with \gpefile{parameters.in}, any changes to this file will require the
  code to be recompiled.

  \subsection{\label{subsec:run.in}run.in}
  This file defines the main parameters for the run, as a set of Fortran
  \verb"namelist"s.  Each namelist loosely collects together related
  parameters.  The namelists are:
  %
  \begin{itemize}
    \item \gpevar{run\_params} --- these are parameters to do with the run
      itself, such as time step, time stepping scheme, when the run should end,
      which equation to solve, etc.;
    \item \gpevar{eqn\_params} --- these parameters set properties of the
      equation, such as whether it should be solved in a moving reference
      frame, trap parameters, random phase parameters, etc.;
    \item \gpevar{io\_params} --- these parameters control input/output, for
      example, what data should be saved and how often;
    \item \gpevar{misc\_params} --- miscellaneous parameters which do not fit
      in the other categories.
  \end{itemize}
  %
  Table~\ref{tab:run.in} describes these parameters in detail.
  %
  \begin{center}
    \begin{longtable}[ht]{llp{0.6\textwidth}}
      \hline
      Parameter & Type & Description \\
      \hline
      \gpevar{tau} & real & The initial timestep to try.  Valid for both
      imaginary and real time. \\
      %
      \gpevar{end\_time} & real & The final (dimensionless) time. \\
      %
      \gpevar{xr} & real & The $x$-coordinate of the right-hand-side of the
      computational box (the left-hand-side is set to -xr). \\
      %
      \gpevar{yr} & real & As above but for the $y$-coordinate. \\
      %
      \gpevar{zr} & real & As above but for the $z$-coordinate. \\
      %
      \gpevar{scheme} & character & The time stepping scheme to use.  This must
      be set to one of \verb"euler" (for explicit second-order Euler time
      stepping), \verb"rk2" (for explicit second-order Runge-Kutta time
      stepping), \verb"rk4" (for explicit fourth-order Runge-Kutta time
      stepping), or \verb"rk45" (for explicit adaptive fourth-order
      Runge-Kutta-Fehlberg time stepping). \\
      %
      \gpevar{eqn\_to\_solve} & integer & The form of the GPE to solve.  See
      \S\ref{sec:nondimgpe} for the possible values to use. \\
      %
      \gpevar{bcs} & integer & The boundary conditions to use.  Set to \verb"1"
      for periodic BCs; set to \verb"2" for reflective BCs. \\
      %
      \gpevar{order} & integer & The order of the derivatives to use.  Set to
      \verb"2" for second-order; set to \verb"4" for fourth-order. \\
      %
      \gpevar{restart} & logical & Set to \verb".true." to do a restart of a
      previous run.  See \S\ref{sec:restart}. \\
      % 
      \gpevar{saved\_restart} & logical & Set to \verb".true." if using
      filtered data from a previous run to multiply with the initial condition
      of a new run.  \\
      %
      \gpevar{renorm} & logical & Set to \verb".true." if the wavefunction
      should be renormalised at every time step in imaginary time. \\
      % 
      \gpevar{imprint\_vl} & logical & Set to \verb".true." if the wavefunction
      should be multiplied by a vortex line at each time step in imaginary
      time. \\
      %
      \gpevar{stop\_imag} & logical & Set to \verb".true." if the run should
      stop at the end of imaginary time, \ie when the relative norms of
      successive time steps are deemed to be sufficiently close (currently
      $10^{-12}$). \\
      %
      \gpevar{real\_time} & logical & Set to \verb".true." if the run should be
      started in real time. \\
      %
      \gpevar{Urhs} & real & Set non-zero to solve the equation in a moving
      reference frame. \\
      %
      \gpevar{diss\_amp} & real & Set non-zero to include dissipation of this
      amplitude at the boundaries. \\
      %
      \gpevar{scal} & real & Set non-zero to scale vortex rings/lines. \\
      %
      \gpevar{nv} & real & For random phase approximation, the total mass per
      unit volume. \\
      % 
      \gpevar{enerv} & real & For random phase approximation, the total kinetic
      energy per unit volume. \\
      % 
      \gpevar{g} & real & Interaction parameter for trapped condensate. \\
      %
      \gpevar{mu} & real & Chemical potential for trapped condensate. \\
      %
      \gpevar{nn} & real & Number of atoms in a trapped condensate.  Should
      currently be left to \verb"1.0", and instead tune \gpevar{mu} and
      \gpevar{g} to specify \gpevar{nn}. \\
      %
      \gpevar{omx} & real & Frequency of trap in $x$-direction. \\
      %
      \gpevar{omy} & real & Frequency of trap in $y$-direction. \\
      %
      \gpevar{omz} & real & Frequency of trap in $z$-direction. \\
      %
      \gpevar{save\_rate} & integer & The rate at which time-series data should
      be saved (roughly corresponding to the number of time steps). \\
      % 
      \gpevar{save\_rate2} & real & How often (in terms of actual time units)
      isosurface data should be saved (3D isosurfaces, 2D surfaces). \\
      %
      \gpevar{save\_rate3} & real & How often (in terms of actual time units)
      data to do with condensed particles and PDFs should be saved. \\
      % 
      \gpevar{p\_save} & real & How often (in terms of actual time units) the
      code should save its own state, so that it can be restarted in the event
      of a machine failure, for example. \\
      % 
      \gpevar{save\_contour} & logical & Should 2D contour data be saved? \\
      %
      \gpevar{save\_3d} & logical & Should 3D isosurface data be saved? \\
      %
      \gpevar{save\_filter} & logical & Should 3D filtered isosurfaces of the
      density be saved? \\
      %
      \gpevar{filter\_kc} & real & The cutoff wavenumber used to filter the
      isosurfaces.  \\
      %
      \gpevar{save\_average} & logical & Should 3D time-averaged isosurfaces of
      the density be saved? \\
      %
      \gpevar{save\_spectrum} & logical & Should various spectra be saved
      (mainly for random phase approximation)? \\
      %
      \gpevar{save\_pdf} & logical & Should PDFs of the velocity components be
      saved? \\
      %
      \gpevar{save\_vcf} & logical & Should the velocity correlation function
      be saved?  \\
      %
      \gpevar{save\_ll} & logical & Should the vortex line length be saved? \\
      %
      \gpevar{save\_zeros} & logical & Should the points of zero density be
      saved?  (Not reliable!) \\
      \hline\hline
      \caption{\label{tab:run.in}Run-time parameters to set.}
    \end{longtable}
  \end{center}

  \section{\label{sec:makefile}Makefile}
  Settings in the Makefile may need to be changed, depending on the
  architecture on which the code is run, and what compilers are available to
  you.  Table~\ref{tab:makefile} describes the Makefile variables.
  %
  \begin{table}[ht]
    \centering
    \begin{tabular}{lp{0.67\textwidth}}
      \hline
      Variable & Description \\
      \hline
      \gpevar{OBJECT} & The name of the executable produced on compilation. \\
      %
      \gpevar{OBJS} & The object files that should be linked. \\
      %
      \gpevar{FC} & The Fortran compiler to be used.  This could be the MPI
      wrapper compiler \verb"mpif90", but the underlying Fortran compiler must
      be able to compile the code (\eg sunf95, ifort, gfortran). \\
      %
      \gpevar{FFLAGS} & Compiler flags.  See the compiler's manual.  Sunf95
      works well with \verb"-fast".  If nonsense results are produced
      \verb"-fsimple=0" might be required. \\
      %
      \gpevar{LDFFTW} & FFTw libraries to link.  Compiling with \verb"make precision=single" will link the single precision libraries. \\
      %
      \gpevar{LDFLAGS} & Any extra flags required by the linker.  If
      \verb"mpif90" is not the compiler, then all the MPI libraries will need
      to be linked. \\
      %
      \gpevar{INCLUDE} & Include path (\eg for MPI header files). \\
      \hline\hline
    \end{tabular}
    \caption{\label{tab:makefile}Description of Makefile variables.}
  \end{table}

  \section{\label{sec:output}Program output files}
  The output that the program produces depends on the parameters set in the
  \gpevar{io\_params} namelist.  Table~\ref{tab:output} briefly describes these
  files.
  %
  \begin{table}[ht]
    \centering
    \begin{tabular}{lp{0.67\textwidth}}
      \hline
      File & Description \\
      \hline
      \gpefile{energy.dat} & Saves the energy at each time. \\
      %
      \gpefile{linelength.dat} & Saves the total line length of vortices in the
      condensate. \\
      %
      \gpefile{mass.dat} & Saves the mass at each time. \\
      %
      \gpefile{minmax\_*.dat} & Saves the minimum and maximum of the density,
      filtered density and time-averaged density over the duration of the run.
      \\
      %
      \gpefile{norm.dat} & Saves the norm at each time. \\
      %
      \gpefile{misc.dat} & Any miscellaneous data can be sent to this file. \\
      %
      \gpefile{momentum.dat} & Saves the three components of the momentum. \\
      %
      \gpefile{p\_saved.dat} & The values of the time index \gpevar{p} when the
      code saved its own state. \\
      %
      \gpefile{save.dat} & The parameters for the most recently saved state. \\
      %
      \gpefile{timestep.dat} & The imaginary and real time step at each time,
      if adaptive time stepping is chosen. \\
      %
      \gpefile{psi\_time.dat} & Saves the real and imaginary time, the real and
      imaginary parts of the wavefunction, the density, and the phase. \\
      %
      \gpefile{proc**} & Numbered directories corresponding to each process
      involved in the run.  Each of these directories contains the binary files
      listed below. \\
      %
      \gpefile{end\_state.dat} & The saved state of the run. \\
      %
      \gpefile{im\_zeros*******.dat} & The coordinates where the imaginary part
      of the wavefunction goes to zero. \\
      %
      \gpefile{re\_zeros*******.dat} & The coordinates where the real part of
      the wavefunction goes to zero. \\
      %
      \gpefile{dens*******.dat} & If 3D isosurfaces are requested, the data for
      them are saved in these files. \\
      %
      \gpefile{filtered*******.dat} & As above, but for filtered data. \\
      %
      \gpefile{ave*******.dat} & As above, but for time-averaged data. \\
      %
      \gpefile{spectrum*******.dat} & The spectrum ($n_{\vec{k}}$ vs.
      $\vec{k}$). \\
      %
      \gpefile{zeros*******.dat} & The coordinates where the real and imaginary
      parts of the wavefunction simultaneously go to zero. \\
      \hline\hline
    \end{tabular}
    \caption{\label{tab:output}Output files from the GPE code.}
  \end{table}

  \subsection{\label{subsec:binary}Binary layout}
  If you do not have access to IDL, but still want to be able to view contour
  and isosurface plots, then you will need to know how this data is arranged in
  the numbered \gpefile{dens*******.dat} files, within each \gpefile{proc}
  directory.  Table~\ref{tab:binary} describes the data which are saved to
  these files, and how much space (in terms of bytes) is needed to store the
  data.
  %
  \begin{table}[ht]
    \centering
    \begin{tabular}{llccp{0.40\textwidth}}
      \hline
      Variable & Type & \multicolumn{2}{c}{Size (bytes)} & Description \\
      & & Single & Double & \\
      \hline
      \gpevar{t+im\_t} & real & 4 & 8 & The total time elapsed. \\
      \gpevar{nx} & integer & 4 & 4 & Number of grid points in the
      $x$-direction. \\
      \gpevar{ny} & integer & 4 & 4 & Number of grid points in the
      $y$-direction. \\
      \gpevar{nz} & integer & 4 & 4 & Number of grid points in the
      $z$-direction. \\
      \gpevar{nyprocs} & integer & 4 & 4 & Number of processes in the
      $y$-direction. \\
      \gpevar{nzprocs} & integer & 4 & 4 & Number of processes in the
      $z$-direction. \\
      \gpevar{js} & integer & 4 & 4 & Starting index of data in $y$-direction,
      local to each process. \\
      \gpevar{je} & integer & 4 & 4 & Ending index of data in $y$-direction,
      local to each process. \\
      \gpevar{ks} & integer & 4 & 4 & Starting index of data in $z$-direction,
      local to each process. \\
      \gpevar{ke} & integer & 4 & 4 & Ending index of data in $z$-direction,
      local to each process. \\
      \gpevar{psi} & complex & \small{8\gpevar{nx*nyl*nzl}} &
      \small{16\gpevar{nx*nyl*nzl}} & Complex wavefunction $\psi$, local to
      each process. \\
      \gpevar{x} & real & 4\gpevar{nx} & 8\gpevar{nx} & The grid array in the
      $x$-direction. \\
      \gpevar{y} & real & 4\gpevar{ny} & 8\gpevar{ny} & The grid array in the
      $y$-direction. \\
      \gpevar{z} & real & 4\gpevar{nz} & 8\gpevar{nz} & The grid array in the
      $z$-direction. \\
      \hline\hline
    \end{tabular}
    \caption{\label{tab:binary}Data saved to the \gpefile{dens*******.dat}
      files, with sizes in bytes for both single and double precision (assuming
      x86 or x86-64 architecture).  The number of grid points local to each
      process in the $y$- and $z$-directions are given by \gpevar{nyl=je-js}
      and \gpevar{nzl=ke-ks}.}
  \end{table}

  \section{run.sh}
  This script can be used to start a run on a shared memory machine, such as
  those with the latest multi-core processors, or older multi-processor
  machines.  Usage instructions are provided with the script itself; run with
  no arguments to see the help.

  \section{\label{sec:gpe.pro}The IDL gpe.pro program}
  If you have access to IDL, then all of the contour and isosurface
  visualisation can be done through the \gpefile{gpe.pro} program.  This
  section will briefly describe its use.  Detailed examples of some aspects of
  the program are given in \S\ref{sec:viewing_results}.

  \subsection{Floating point precision}
  It is important to note the precision of the run for which you want to view
  results.  If you have performed a double-precision run, then all \gpevar{gpe}
  commands must include the \verb"/dbl" keyword.

  \subsection{Fortran unformatted I/O}
  The GPE code used to write binary data using Fortran 77 unformatted,
  record-based I/O.  With the advent of Fortran 2003 stream I/O, this is no
  longer necessary, and record markers are not part of the data.  It might
  still be necessary to view the old data, however, in which case the
  \verb"/f77" keyword should be used in all \gpevar{gpe} commands.

  \subsection{Isosurface plots}
  The program will produce an isosurface plot of the density $\abs{\psi}^{2}$
  if no keywords or options are provided, \eg
  %
  \begin{Verbatim}
    gpe, 0, 0
  \end{Verbatim}
  %
  To change the default surface colour, set the \verb"index" option, \eg
  %
  \begin{Verbatim}
    gpe, 0, 0, index=100
  \end{Verbatim}
  %
  The \verb"index" option must be in the range 0 to 255, and corresponds to the
  index into the currently loaded IDL colour table.
  %
  The isosurface level can also be controlled, by using the \verb"level"
  option, \eg
  %
  \begin{Verbatim}
    gpe, 0, 0, level=0.4
  \end{Verbatim}

  \subsection{Contour plots}
  A contour plot of the density in the $(x,y)$-plane at $z=0$ is displayed with
  the addition of the \verb"/cntr" keyword, \eg
  %
  \begin{Verbatim}
    gpe, 0, 0, /cntr
  \end{Verbatim}
  %
  Plots of the phase or velocities can also be produced by adding the
  \verb"/phase", \verb"/vx", \verb"/vy", or \verb"/vz" keywords, \eg
  %
  \begin{Verbatim}
    gpe, 0, 0, /cntr, /phase
  \end{Verbatim}
  %
  The position of the contour slice can be controlled by using the \verb"xpos",
  \verb"ypos", or \verb"zpos" options, \eg
  %
  \begin{Verbatim}
    gpe, 0, 0, /cntr, xpos=2.0
  \end{Verbatim}
  %
  The position is given in real units (as opposed to grid units).
  %
  The plane in which the contour slice sits can be controlled with the
  \verb"dir" option, \eg
  %
  \begin{Verbatim}
    gpe, 0, 0, /cntr, dir='y'
  \end{Verbatim}
  %
  This will produce a contour plot in the $(x,z)$-plane.

  \subsection{Slice plots}
  One-dimensional slices through the data can also be generated by using the
  \verb"/slice" keyword, and the position and direction controlled as for
  contour plots, \eg
  %
  \begin{Verbatim}
    gpe, 0, 0, /slice, xpos=3.4, dir='z'
  \end{Verbatim}

  \subsection{Contour animations}
  A series of contour snapshots can be generated by using the \verb"/c_anim"
  keyword, \eg
  %
  \begin{Verbatim}
    gpe, 0, 9, /cntr, /phase, /c_anim
  \end{Verbatim}
  %
  Snapshots are saved to the directory \gpefile{images}, from the directory
  under which IDL is started, so this directory must exist before attempting to
  create snapshots, otherwise an error will result.

  \subsection{Isosurface animations}
  A series of isosurface snapshots can be generated by using the \verb"/png"
  keyword, \eg
  %
  \begin{Verbatim}
    gpe, 0, 9, /png
  \end{Verbatim}
  %
  Currently, it is only possible to generate isosurfaces of the density.

  \subsection{EPS output}
  High quality EPS figures of all 1D, 2D, and 3D plots can be produced, by
  using the \verb"/eps" keyword.  Figures are saved in the images directory (as
  for the snapshots in the animations above), so this directory must exist
  prior to attempting to save as EPS.

\end{chapter}
